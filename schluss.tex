\section{Schlussfolgerungen}
\onlydraft{
Überblick über die Bewertung der Systeme
Überblick über die Eignung für den Produktiveinsatz
Ausblick für weitere Entwicklung der VM-Verschiebung
}

Keines der betrachteten Systeme kann die gestellen Anforderungen in
vollem Umfang erfüllen. Gleichwohl bietet die Kombination der von IBM
vorgestellten Techniken, insbesondere durch die Implementierung der
vorgeschlagenen Standards, schon ein hohes Maß gut handhabbarer und
konfigurierbarer Live-Migration von \acp{VM}. Während sich die
Lösungen von VMWare eher in Richtung dynamischer Anpassung von
\ac{VM}-Infrastruktur orientieren, sind die von HP bereitgestellten
Techniken eher auf Ausfallsicherheit und eine sehr hohe Verfügbarkeit
ausgelegt. Beide lassen aber bei der Interoperabilität miteinander und
auch mit anderen Live-Migrationssystemen noch Spielraum. Es
ist zu wünschen, dass sich auf Standards dafür geeinigt wird,
gegebenenfalls auch die von IBM vorgeschlagenen. Die \ac{ldom}-Lösung von
Oracle hat einen hohen Hardwarebezug, wodurch der Vergleich mit
anderen Anbietern schwierig ist. 

Momentan bietet es sich an, wenn schon eine Virtualisieungslösung
einer der Anbieter im Einsatz ist, dessen Live-Migrationsfähigkeiten
zu prüfen und zu nutzen. Zum aktuellen Zeitpunkt ist es jedoch nicht
möglich, eine allgemeine Empfehlung für oder gegen eine bestimmte
Lösung auszusprechen. Das Wechseln der Virtualisierungslösung nur zum
Zweck, die entsprechende Live-Migrationsfähigkeit zu nutzen, lohnt
momentan nicht.

In Zukunft ist mit einer weitergehenden Standardisierung der an
Live-Migration beteiligten Techniken zu Rechnen, die Anbieter scheinen
auch daran Interesse zu haben. Es ist damit zu rechnen, dass
Live-Migration, genau wie Virtualisierung, eine Normalität beim
Serverbetrieb werden wird.



%%% Local Variables: 
%%% mode: latex
%%% TeX-master: "FelgentreffPape_2010_Live-MigrationInVirtuellenUmgebungen"
%%% End: 
