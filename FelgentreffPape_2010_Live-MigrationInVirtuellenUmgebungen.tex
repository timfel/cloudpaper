%\documentclass[a4paper,conference,compsoc]{IEEEtran}
\documentclass[draft,journal]{IEEEtran}
%\documentclass[a4paper]{IEEEtran}

\makeatletter
\def\markboth#1#2{\def\leftmark{\@IEEEcompsoconly{\sffamily}\MakeUppercase{\protect#1}}%
\def\rightmark{\@IEEEcompsoconly{\sffamily}\MakeUppercase{\protect#2}}}
\makeatother

%%
%% enable searchable PDFs
%%
\usepackage{cmap}
%%
\usepackage[T1]{fontenc}
\usepackage[utf8]{inputenc}
\usepackage{textcomp,textcase}
\usepackage[ngerman]{babel}

\usepackage{cite}
\usepackage[caption=false,font=footnotesize]{subfig}
\usepackage{graphicx}

\usepackage{float,mparhack,fixltx2e,stfloats}

\usepackage{hyperref}

\usepackage{xspace}
\usepackage{listings}
\usepackage{url}

%%
%% spaced (small) caps
%% 
\DeclareRobustCommand{\spacedallcaps}[1]{\textls[160]{\MakeTextUppercase{#1}}}%
\DeclareRobustCommand{\spacedlowsmallcaps}[1]{{\MakeTextLowercase{\scshape #1}}}%
\DeclareRobustCommand\Largespacedlowsmallcaps[1]{\spacedlowsmallcaps{\Large #1}}

\errorcontextlines=909

%% setups
\fnbelowfloat
\usepackage[final]{microtype}
\microtypesetup{babel,stretch=10,shrink=15,step=3,tracking=smallcaps}

\lstnewenvironment{code}[1][]%
{\lstset{#1}\relax%
}{%
}

%\usepackage{style}
\usepackage{blindtext}
\bibliographystyle{IEEEtran}

\begin{document}
\title{Live-Migration in virtuellen Umgebungen\\
  \large{Hypervisor - What have you done for me today?}
}
\author{Tim~Felgentreff und~Tobias~Pape%
\thanks{%
  Tim Felgentreff und Tobias Pape studieren am
  Hasso-Plattner-Institut, Potsdam.\goodbreak
  Email: \{vorname.nachname\}@student.hpi.uni-potsdam.de}%
}
\markboth{Industrieseminar Cloud-Computing}{Felgentreff, Pape: Live-Migration}

\maketitle
\tableofcontents

\begin{abstract}
  
\end{abstract}

% \begin{IEEEkeywords}
% \end{IEEEkeywords}
\IEEEpeerreviewmaketitle

\section{Einführung}
\label{sec:einfuehrung}
% \IEEEPARstart{H}{eutzutage} gibt es zu viele Probleme
% Today, many organizations deploy a multitude of services on a large
% number of servers. In some cases, all these services run on different
% machines not because one machine cannot handle the load, but for
% reasons of reliability.  All software is faulty~\cite{zellerprograms},
% and operating systems with thousands of lines of code, running servers
% written by different software vendors simply cannot be trusted to
% provide uptimes of 99.9 percent or more. Putting each service on a
% separate machine will achieve reliability at the expense of
% maintainability simply because a large number of physical machines is
% involved.

% Virtual machine technology has been around for more than 40
% years~\cite{tanenbaum1992modern}. The technology to host multiple
% machines on a single system reduces the space and power requirements
% for data centers and represents enormous cost savings for companies
% like Amazon, Microsoft or Google, which deploy hundreds or thousands
% of servers for a variety of services. Unsurprisingly, virtualization
% for cost reduction and transparent scalability is becoming
% increasingly popular. Cloud providers such as Amazon, Rackspace or
% Engine Yard offer scalable hosting solutions for companies and
% individuals who cannot of do not want to maintain a private cloud.

% There are obvious issues with isolation, however. As a service
% provider, I want to avoid downtime as much as possible. Thus, I cannot
% easily move my virtual machine from one provider to another. I cannot
% easily migrate existing VMs to new a hypervisor architecture or a
% completely new virtualization solution. If hardware is failing, I
% cannot simply migrate the VM to new hardware without taking it down.

% Asger Jensen and Dr. Jacob Gorm Hansen proposed live VM migration as a
% solution to these problems. They were the first to demonstrate live
% migration of a running VM in 2002 on a research hypervisor called
% NomadBIOS. 
- Firmen sehen heutzutage so aus
- Cloud könnte für sie interessant sein
- Welche Rolle Live-Migration da für sie spielt, und wie es ihnen
helfen kann, zeigen wir

\subsection{Der Status in Rechenzentren}
- Derzeit Rechner im Rechenzentrum
- Cluster
- Grid
- Probleme:
  - Ungleichmäßige Auslastung
  - Wartungskosten (Admins)
  - Downtime bei Upgrades/HW-Problemen
  - Operationskosten (Strom)

\subsection{Die Cloud als Ziel}
\label{sec:sota}
- Public vs Private
Für potentielle Nutzer einer Cloud Lösung präsentieren sich derzeit
zwei Möglichkeiten: Eine eigene, business-interne Cloud Plattform zu
installieren, oder 

In this part we will explore the state of the art in live VM
migration, both from a technical as from a use-case point of view.
Ich will scalieren:
\begin{itemize}
\item Public Cloud
\item Private Cloud
\end{itemize}
- Um zu entscheiden zunächst wissen, {\bf was} in die cloud geht

\subsubsection{Level der Virtualisierung}
\label{sec:def-virtualisierung}
\url{http://www.dcl.hpi.uni-potsdam.de/teaching/cloudIndustSem/slides/giesekus.pdf}
\begin{itemize}
\item Level der virualisierung
  \begin{itemize}
  \item hw (ldoms)
  \item hypervisoren !!
    
    Virtuelle Maschinen (VMs) laufen auf physikalischen Maschinen durch
    einen \emph{Hypervisor}, ein minimales Betriebssystem, dass den
    Zugriff auf die physikalische Hardware regelt und Routing zwischen dem
    physikalischen und virtuellen Netzwerk
    bereitstellt~\cite{tanenbaum1992modern}. Wenn eine VM ausfällt,
    bleiben die anderen davon unbeeinflusst. Da des weiteren alle
    Operationen innerhalb einer VM virtualisiert sind, kann der Hypervisor
    den Zustand eines virtuellen Servers in regelmäßigen Abständen
    sichern, und so bei Ausfall die VM von einer früheren Sicherung
    wiederherstellen.

  \item Os-partitioning (Solaris Zones)
  \item Process VMs (JVM etc.pp.)
  \end{itemize}
\end{itemize}

\subsubsection{Was geht in die Cloud?}
Auf welchen Ebenen gibt es "`Clouds"', welche Ebene betrachten wir,
PaaS vs IaaS.

\subsection{Neue Probleme in Clouds}
Probleme:
\begin{itemize}
\item Public
  \begin{itemize}
  \item Bindung an Provider
  \item Scalierbedarf zunächst unklar -> Kostenfrage
  \item Physikalischer Ort der Daten unbekannt -> VMs müssen in die
    private Cloud oder in Rechenzentren verschoben werden können ->
    gesetzl. Vorschriften, Intellectual Property, \ldots
  \end{itemize}
\item Private
  \begin{itemize}
  \item Scalierbedarf gegen Wartungsaufwand -> Menge der HW
  \item Gleichmäßige Auslastung zunächst genauso schwierig wie vorher
  \item Hardware-Wechsel soll nicht mehr zu Donwtime führen
  \item Virtualisierung 
  \end{itemize}
\end{itemize}

\section{Live-Migration als Teil der Lösung}
\label{sec:livemigration}
Dr. Jacob Gorm Hansen pr"asentierte das NomadBIOS~\cite{needed} als
Live-Migrations-L"osung als Teil der L"osung f"ur oben genannte
Probleme der Clouds. Im Folgenden werden wir auf die von ihm
vorgestellte L"osung technische rekapitulieren und dabei den
L"osungsbeitrag aufzeigen.

\subsection{Verlässlichkeit von Virtualisierung}
In jedem verteilten System muss man mit der Realität umgehen, dass
Hardware und Software Fehler enthält, die zu Ausfällen führen
können. Da scheint die Idee, viele Virtuelle Maschinen auf wenigen
physikalischen Systemen auszuführen zunächst wie ein Schritt in die
falsche Richtung: Wenn dann ein physikalischer Server ausfällt, oder
die Virtualisierungssoftware einen Fehler enthält, fallen potentiell
eine ganze Reihe von Diensten auf einmal aus. In der Realität
entstehen die meisten Fehler in Datencentern allerdings durch
Softwarefehler~\cite{tanenbaum1992modern}. Softwarefehler können umso
häufiger zutage treten, je mehr unterschiedliche Systeme miteinander
interagieren müssen, und je größer die Code-Basis der einzelnen
Systeme ist~\cite{zellerprograms}. Durch Virtualisierung kann man
Softwaresysteme voneinander isolieren, ohne für jedes System einen
eigenen, physikalischen Server mit eigenem Betriebssystem
bereitzustellen. Der Hypervisor selbst ist in solch einem Fall die
einzige Software, die im Kernel Modus läuft, und dieser hat um
Größenordnungen weniger Zeilen Quelltext als ein aktuelles
Betriebssystem~\cite{tanenbaum1992modern}. Folglich enthält der
Hypervisor wahrscheinlich weniger Bugs~\cite{zellerprograms}.

Kritische Dienste können so in kompletter Isolation ausgeführt um das
Risiko eines Ausfalls durch inkompatible Software auf demselben Server
zu minimieren.

Nichtsdestotrotz müssen auch Vorkehrungen gegen Hardwarefehler
getroffen werden. In klassischen Hosting-Umgebungen konnte man bei
immanenten Ausfällen von physikalischen Systemen relevante Prozesse
auf neue VMs migrieren~\cite{hansen2004self}. Für eine gegebene VM
konnte eine Ähnliche (z.B. aus einem früheren, gesicherten Zustand)
auf einem neuen Server gestartet werden. Der Zustand der relevanten
Prozesse wurde soweit als möglich gemäß den Gegebenheiten des
Betriebssystems kopiert und neue Dienstanfragen wurden dann auf den
neuen Server umgeleitet. Die alte VM musste allerdings zunächst online
bleiben, um Abhängigkeiten von Dienstanfragen bereitzustellen, die
sich noch in Ausführung befanden, als der Hardwarealarm ausgelöst
wurde~\cite{clark2005live}.

Diese Vorgehensweise zur Migration auf Prozessebene hat also den
wesentlichen Nachteil, dass laufende Prozesse nicht verschoben werden
können, sondern noch auf der bestehenden Hardware zum Ende laufen
müssen. Im Falle eines Web-Servers, bei dem jeder Request auch an
einen neuen Prozess gehen kann, ist das nicht weiter
problematisch. Schwieriger ist die Situation bei Diensten wie
z.B. Online-Spieleplattformen, bei denen ein Serverprozess
TCP-Verbindungen zu mehreren Clients teilweise über Stunden aufrecht
erhalten muss. Wenn ein Hardwareausfall kurz bevor steht, bleibt hier
meist nichts anderes übrig, als das Spiel zu unterbrechen: Das kann
bedeutet mindestens unzufriedene Kunden, und vielleicht sogar
verringerten Umsatz.

Mit dem von Dr. Gorm Hansen vorgestellten NomadBIOS kann auf Hardwarefehler
angemessener reagiert werden: Zuerst kann die VM bis aufs letzte
Detail vom aktuellen Stand kopiert werden, das heißt, dass Änderungen,
die seit der letzten Checkpoint-Sicherung geschehen sind, nicht
verloren gehen. Wichtiger, allerdings, ist, dass auch Prozesse, die
sich gerade in Ausführung befinden, migriert werden können. Im
Beispiel der Spieleplattform bedeutet dies, dass Spiele nicht
unterbrochen werden müssen, sondern, für den Nutzer transparent, auf
den neuen Server migriert werden.

\subsection{Interoparibilität von Virtualisierungs-Lösungen}
F"ur Unternehmen, die mit "offentlichem Cloud-Hosting lieb"augeln,
kann die Bindung an ein bestimmtes System schwerwiegende Konsequenzen
nach sich ziehen. Um mit der Flexibilit"at vom eigenen Rechenzentrum
mitzuhalten, darf die Virtualisierungstechnologie des Providers nicht
die Verwendung des Systems innerhalb das Unternehmens
einschr"anken. Insbesondere muss f"ur den Kunden die M"oglichkeit
bestehen, auf eine andere Virtualisierungstechnologie umzusteigen,
vorhandene VMs direkt in die Cloud zu bewegen oder auch VMs aus der
Cloud wieder in das eigene Rechenzentrum oder auf die Rechner der
Entwickler umzuziehen. Die Interoparibilität der verwendeten
Virtualisierungsl"osungen spielt hier eine wichtige Rolle.

Virtuelle Maschinen variieren stark zwischen verschiedenen
Virtualisierungsl"osungen. Das f"angt mit den Formaten der virtuellen
Dateisysteme an~\cite{none} und geht bis hinunter zu den spezifischen
Arten der virtualisierten Hardwarekomponenten~\cite{none}. Das
bedeutet im Einzelnen, dass der Wechsel zwischen Hypervisor
Technologien von au"sen eine Konvertierung zwischen den (oft
propriet"aren) Formaten bedeutet, aber auch im virtualisierten Host
unter Umst"anden Treiber neu installiert werden m"ussen. Solch
tiefgreifende "Anderungen, die oftmals gezwungenerma"sen mit Neustarts
der virtuellen Betriebssysteme einhergehen, sind kaum ohne
Ausfallzeiten zu unternehmen. Folglich hat ein Cloud-Hosting Provider
allein durch die Auswahl seiner Virtualisierungsl"osung die
M"oglichkeit, Kunden sehr stark zu binden.

Live-Migration verspricht hier Abhilfe, wenn nur die
Migrationsprotokolle der Hypervisoren kompatibel
sind~\cite{none}. Dann n"amlich kann, ohne Umweg "uber die
serialisierten Formate der Virtualisierungsl"osungen, das
virtualisierte System direkt zwischen zwei VMs kopiert werden, und
sozusagen zur Laufzeit konvertiert werden.

\subsection{Inter-Cloud Verschiebungen}
\label{sec:movclouds}
Probleme durch inkompatible Virtualisierungen werden im Rahmen des
Outsourcings in die \emph{Public Cloud} noch verst"arkt, da die
Entscheidung f"ur einen Cloud-Provider unter Umst"anden bedeutet, dass
die Dienste nur unter erheblichem Aufwand wieder in eine andere Cloud
verschiebbar sind. Das bedeutet f"ur den Cloud-Kunden eine starke
Bindung an den Hosting Provider, eine Bindung die f"ur Unternehmen so
problematisch sein kann, dass auf das Hosting bei einem fremden
Provider ganz verzichtet werden muss.

Prozessmigrationssysteme, wie sie in den 1980er Jahren entwickelt
wurden, wie z.B. das Sprite oder das MOSIX
System~\cite{hansen2004self} benutzten F"ahigkeiten des Hypervisors
sowie der darauf laufenden Betriebssysteme in Konjunktion und waren
damit per-se nicht auf beliebigen Kombinationen der beiden
ausf"uhrbar. Desweiteren schr"ankte die Notwendigkeit von spezifischen
Anpassungen and verschiedenen Teilen des Softwarestacks und
propriet"are Protokolle die Verwendung derselben "uber die Grenzen von
Unternehmen und Organisationen hinweg ein. Migration ganzer VMs bietet
hier einen Ausweg durch kleinere Schnittstellen und Wegfallen der
Kommunikation zwischen den migrierten Systemen nachdem die neue VM in
Betrieb geht. Da kein kompliziertes Routing zwischen der neuen und der
alten VM vonstatten gehen muss, beschr"ankt sich die Kommunikation
(und damit der Anspruch an die Schnittstellen der Providers) auf die
initiale Synchronisationsphase.

Selbst wenn der Migrationspfad zwischen Virtualisierungsl"osungen
vorhanden ist, ist das Bewegen einer VM aus einer fremden Cloud
trotzdem oft nicht trivial. Cloud Provider bieten derzeit noch
verschiedene, teilweise inkompatible und in ihrer M"achtigkeit
schwankende Zugriffsm"oglichkeiten auf die gehosteten Systeme. Kleine
Provider wie Heroku~\cite{none} bieten "ublicherweise nur vollen
Zugriff auf einen Teil der gehosteten Software und erlauben
Maschinenkonfiguration nach dem Baukastenprinzip. Ein vollst"andiges
VM Image ist so einfach nicht zu bekommen. Im Vergleich dazu bieten
Engine Yard~\cite{none} und VMWare ausgefeilte Tools um VMs mit
Meta-Konfigurationen~\cite{none} vollst"andig selbst einzurichten und
bei Problemen auch komplette Images dieser Maschinen zu
erhalten~\cite{none}. Wieder muss allerdings gesagt werden, dass diese
Interfaces inkompatibel sind, und es nicht ohne Ausfallzeit m"oglich
sein d"urfte, Dienste von einer in
die andere Cloud zu bewegen.\\

Es gibt Bem"uhungen diese Verschiedenen Aspekte des Cloud Hostings zu
standardisieren~\cite{cloudstandard}. Darin inbegriffen sind die
Beschreibungen von VMs sowie die APIs die Hypervisoren
bereitstellen. Dieser Prozess ist allerdings noch nicht abgeschlossen,
und obwohl gro"se Hosting-Anbieter wie Rackspace, VMWare und Engine
Yard daran teilnehmen~\cite{none}, fehlt z.B. noch Amazon als der
gr"o"ste Hosting-Provider in Europa.

Die Fortentwicklung der VM-Migrationstechnologie, wie sie
Dr. Gorm Hansen beschrieben hat, k"onnte eine einfache L"osung f"ur
dieses Problem bieten, indem sie die F"ahigkeiten zur Live-Migration
vollst"andig in das Betriebssystem verschiebt. Mittels
standardisierter Hypervisor APIs k"onnte dann vom virtuellen System
aus der Migrationsprozess angesto"sen werden, und der Umzug in eine
neue Cloud auch ohne Mitwirken des Hosting-Providers m"oglich
werden. Das bedeutet effektiv eine Reduzierung der Bindung an eine
bestimmte Cloud und vermindert das Risiko von \emph{Public Clouds} aus
Sicht des Unternehmes~\cite{none}.

\subsection{Hardware-Aufstockung ohne Ausfallzeigen}
Moderne IT-Unternehmen m"ussen auf Markt"anderungen, die Skalierung
erfordern, schnell und mit geringen Kosten reagieren k"onnen. Neue
M"arkte werden oft erst erschlossen, indem Dienste auf g"unstiger
Hardware erpropt werden~\cite{tanenbaum1992modern}. Wenn sich ein
Marktsegment als lukrativ erweist, muss man in der Lage sein, schnell
die Anzahl und Menge der Server zu skalieren. Diese F"ahigkeit ist
entscheidend f"ur den Erfolg eines modernen Web-Unternehmens. Studien
zeigen, dass gerade am Anfang eines Online-Angebots lange Ladezeiten
die Adoption beeintr"achtigen~\cite{kohavi2007online}.

Oftmals k"onnen diese Vorrausetzungen nur durch gro"sz"ugige Planung
der kommendenen Hardwareanforderungen erf"ullt werden, was mit einem
gr"o"seren finanziellen Risiko einhergeht. Gerade kleine Unternehmen
und \emph{Start-Ups} k"onnen sich dies oft nicht leisten. Die
F"ahigkeit, Services in der Cloud anzubieten und \emph{bei Bedarf}
mehr gleichartige VMs zu starten, oder VMs auf st"arkere physikalische
Nodes zu verschieben, ohne daf"ur Downtime in Kauf nehmen zu m"ussen,
kann den Unterschied zwischen einem erfolgreichen Service und einer
Seite, zu der Kunden nicht mehr zur"uckkehren, weil sie zu langsam
ist, machen.

Der Cloud-Hosting Provider \emph{Heroku} hat sich auf ebenso solche,
kleine, Webdienste spezialisiert, und versteckt den Prozess des
Skalierens f"ur den Kunden vollst"andig hinter einem einfach
Schieberegler im Web-Interface: Der Leistungsindex wird hochgeschraubt
und Heroku k"ummert sich um das Spawning und Verschieben von VMs.

In Relation hierzu sind auch Services wie die Google-App Engine zu
nennen, wenngleich sie eher \emph{Infra-Structure as a Service}
(IaaS) zuzurechnen sind, bieten auch sie kleineren Unternehmen die
M"oglichkeit, nur den Umfang an Leistung zu zahlen, der auch
tats"achlich genutzt wird, und bei Bedarf mehr Leistung
"`einzukaufen"'.

\subsection{Adaptive Auslastung}
\label{sec:adaptive-auslastung}
Neben der Notwendigkeit, bei Marktwachstum schnell skalieren zu
k"onnen, gibt es auch die Anforderung an viele Systeme, in
Sto"szeiten, oder bei unvorhergesehenen Nachfrageschwankungen adaptiv
skalieren zu k"onnen. Ein Beispiel nannte \cite{none} von der
Deutschen Bahn, der darauf hinwies, dass gerade wetterbedingte
Ausf"alle die Rechenzentren der Deutschen Bahn oft auf eine harte
Probe stellen, da durch diesen unvorhersehbaren Umstand viele Reisende
zus"atzliche Erkundigungen einziehen wollen. In solch einem Fall ist
es f"ur den Dienstanbieter wichtig, weiterhin einen stabilen und
schnellen Service zu leisten, um die Kundenzufriedenheit nicht
st"arker zu gef"ahrden und gar Kunden zu verlieren. In klassischen
Rechenzentren kann man, wie im Vortrag von \cite{benchmarkingthecloud}
besprochen, die verf"ugbare Kapazit"at auf die erwarteten
Maximalauslastungen optimieren. Das ist jedoch tendentiell teuer und
gerade in Situationen, wo die maximale Abweichung von der
durchschnittlichen Auslastung sehr gro"s ist, oder Spikes sehr selten
auftreten, ist die Investition oft nicht gerechtfertigt.

VM-Live-Migration kann auch hier zum schnellen Verschieben von
Maschienen zwischen leistungsstarken und -schwachen physikalischen
Nodes genutzt werden.

Ein Gegenargument, dass man f"ur diesen Use-Case im Besonderen
hervorbringen kann, ist, dass VM-Migration mit hoher Netzwerklast
einhergeht. In Situationen, in denen der Durchsatz eh gef"ardet ist,
kann man argumentieren, dass die zus"atzliche Netzwerklast durch eine
laufende Migration den Dienst vor"ubergehend ganz zum Erliegen bringen
k"onnte. Die erh"ohte Last liegt, je nach Anwendung, f"ur mehrere
Minuten auf dem Netzwerk, da die Umzugszeiten von der Netzwerklatenz
zwischen den Servern abh"angen. Zu l"osen w"are dieses Problem durch
ein komplett parallel betriebenes Netzwerk, dass speziell f"ur diesen
Anwendungsfall freigehalten wird.

Wir argumentieren, dass die Kosten f"ur einen solchen Parallelbetrieb
zweier Netzwerke weit hinter den Kosten f"ur das vorhalten von
ausreichender, aber meist ungenutzter, Rechenkapazit"at
zur"uckbleiben.

\subsection{Keine Ausfallzeit durch Verschiebungen}
\label{sec:keine-ausfallzeit}
Neben den Use-Cases zur Skalierung und Aufrechterhaltung von
(Cloud-)Services, gibt es ein weiteres Problem, mit dem international
agierende Anbieter umgehen m"ussen, und das sind die sich wandelnden
gesetzlichen Vorschriften zur Datenverarbeitung, die gerade in
Deutschland noch vielfach in Ver"anderung begriffen sind~\cite{none}.

Ein Beispiel ist die von \cite{MrBlah} im Vortrag zu \cite{foo}
genannte Vorschrift, dass personenbezogene Daten von deutschen
Beh"orden nur auf deutschem Boden verarbeitet werden d"urfen. Diese
Art von Gesetz schlie"st Dienst, die in anderen L"andern gro"se
Datenverarbeitungsdienste bieten, oft von Auftr"agen deutscher
Beh"orden aus. Rechenleistung in anderen L"andern als Deutschland ist
unter Umst"anden wesentlich billiger f"ur ein Unternehmen mit einem
hohen Leistungsbedarf. Dauerhaft ein Rechenzentrum in Deutschland zu
betreiben, f"ur die M"oglichkeit, ab und an einen Auftrag von einer
deutschen Beh"orde zu erhalten, wiegt unter Umst"anden nicht den
Ertrag solcher Auftr"age auf.

Durch die M"oglichkeit, ohne Verlust des aktuellen Dienstzustandes die
Virtuelle Maschine umzuziehen, ist es jedoch m"oglich, solchen
gesetzlichen Vorschriften nichtsdestotrotz zu gen"ugen. Indem der
Datenverarbeiter bei einem Cloud Hosting Service auf deutschem Boden
Rechenzeit einkauft, kann er den gestzlichen Vorschriften gerecht
werden. Die VMs, die f"ur den Auftrag bereitgestellt sind, k"onnen
einfach in die Cloud auf deutschem Boden migriert werden. Migrationen
"uber L"andergrenzen hinweg, z.B. durch den Atlantik, sind zwar
notwendigerweise langsamer als zwischen Rechenzentren mit h"oherer
Lokalit"at, bei gro"sen, lang laufenden Auftr"agen ist das jedoch
unter Umst"anden zu vernachl"assigen. 

Auf diese Weise ist es Datenverarbeitungsdiensten m"oglich,
landesspezifischen Einschr"ankungen bez"uglich der geografischen Lage
der physikalischen Nodes gerecht zu werden, und Services "`on-demand"'
anzubieten, ohne die Flexibilit"at von lokal vorbereiteten Maschinen
aufzugeben.

Die Amazon EC2 bietet diese M"oglichkeit bereits zum Teil, indem man
solche Einschr"ankungen beim Kauf der Rechenzeit
angibt~\cite{none}. Derzeit m"ussen dazu allerdings die VMs neu
gestartet werden. Mithilfe von Live-Migration k"onnte man hier einen
Service anbieten, bei dem es m"oglich ist, zur Laufzeit solche
Einschr"ankungen zu konfigurieren, ohne Downtime in Kauf nehmen zu
m"ussen.

\subsection{Langzeit-Lauffähigkeit}
\label{sec:langz-lauff}
Das Versprechen der Cloud ist, Informationen und Dienste jederzeit
verf"ugbar zu haben, unabh"angig von der physikalischen Verf"ugbarkeit
einzelner physikalischer Server oder (Teil-)Netzwerke. Der Aufwand und
damit die Kosten, die durch die Umstellung bestehender Systeme (und
der zugeh"origen Prozesse) entstehen, werden gerechtfertigt dadurch,
dass der Zugang zu kritischen Diensten und Informationen nicht alle
paar Monate unterbrochen wird, um umzugehen mit den Folgen von:
\begin{itemize}
\item Hardware-Ausf"allen
\item Turnus-m"a"sigen Hardware-Upgrades
\item Schwankenden Anforderungen
\item Ausweitung des Kerngesch"afts
\item Hosting Wechsel
\end{itemize}

Wie wir gezeigt haben, kann diesen Problemen mit VM-Level
Live-Migration Technologie begegnet werden. Folglich ist die
Flexibilit"at von Cloud Hosting in Kombination mit dieser Technologie,
im Vergleich zum klassischen Rechenzentrum, in diesen Bereichen
h"oher, und besser an wechselnde Gegebenheiten anzupassen. Ein
Anbieter kann daher l"angerfristig davon ausgehen, dass Dienste und
Informationen verf"ugbar sind. Das vermindert den Druck und die
Kosten, Dienste "`In-House"' und Datensicherungen auschlie"slich "`vor
Ort"' vornehmen zu m"ussen. Gesch"aftsmodelle k"onnen daraufhin f"ur
gr"o"sere Marktschwankungen ausgelegt werden, ohne die Kosten f"ur
Hosting und Hardware unn"otig in die H"ohe zu treiben. Weiterhin
k"onnen strategische Entscheidungen langfristiger getroffen werden, da
die dauerhafte Verf"ugbarkeit einfacher zu gew"ahrleisten ist, und
nicht mit l"angerer Laufzeit rapide schwieriger und damit teurer wird.

\section{Live-Migration-Systeme}
Im Folgenden wollen wir einige konkrete Implementierungen der oben
vorgestellten Techniken vorstellen, und deren Limitationen und
Einsatzbarkeit f"ur die oben genannten Use-Cases betrachten.

\subsection{NomadBIOS and vMotion}
\label{sec:nomadbios--vmware}
Wie im Vortrag von Dr. Gorm Hansen beschrieben, war die erste
vollst"andige Implementierung zur Live-Migration einer kompletten VM
der NomadBIOS Hypervisor.

Das NomadBIOS bietet VM Migration auf dem Hypervisor Level.

\ldots technischer "Uberblick, Diagramme, Netzwerklast, Hypervisor vs
VM-Level, Fokus, Anwendungsbeispiel "`Cloudbewegung"' \ldots

\subsection{IBM}
Xen, active research to enable VM migration on a number of open-source
virtualization solutions (VirtualBox, KVM + Qemu)

\subsection{HP}
HP-UX Migration
OS-Level

\subsection{Sun/Solaris}
Solaris Container

\subsection{Vergleich}
Vergleich der Eignung als Lösung für die oben genannten Probleme
zwischen dem von Dr. Gorm Hansen vorgestellen System und den Alternativen.

\section{Schlussfolgerungen, Zusammenfassung}
Überblick über die Bewertung der Systeme
Überblick über die Eignung für den Produktiveinsatz
Ausblick für weitere Entwicklung der VM-Verschiebung

% trigger a \newpage just before the given reference
% number - used to balance the columns on the last page
% adjust value as needed - may need to be readjusted if
% the document is modified later
%\IEEEtriggeratref{8}
% The "triggered" command can be changed if desired:
%\IEEEtriggercmd{\enlargethispage{-5in}}
\bibliography{IEEEabrv,Live-MigrationInVrituellenUmgebungen}
\end{document}

%%% Local Variables: 
%%% mode: latex
%%% TeX-master: t
%%% End: 
