%\documentclass[a4paper,conference,compsoc]{IEEEtran}
\documentclass[draft,journal]{IEEEtran}
%\documentclass[a4paper]{IEEEtran}

\makeatletter
\def\markboth#1#2{\def\leftmark{\@IEEEcompsoconly{\sffamily}\MakeUppercase{\protect#1}}%
\def\rightmark{\@IEEEcompsoconly{\sffamily}\MakeUppercase{\protect#2}}}
\makeatother

%%
%% enable searchable PDFs
%%
\usepackage{cmap}
%%
\usepackage[T1]{fontenc}
\usepackage[utf8]{inputenc}
\usepackage{textcomp,textcase}
\usepackage[ngerman]{babel}

\usepackage{cite}
\usepackage[caption=false,font=footnotesize]{subfig}
\usepackage{graphicx}

\usepackage{float,mparhack,fixltx2e,stfloats}

\usepackage{hyperref}

\usepackage{xspace}
\usepackage{listings}
\usepackage{url}

%%
%% spaced (small) caps
%% 
\DeclareRobustCommand{\spacedallcaps}[1]{\textls[160]{\MakeTextUppercase{#1}}}%
\DeclareRobustCommand{\spacedlowsmallcaps}[1]{{\MakeTextLowercase{\scshape #1}}}%
\DeclareRobustCommand\Largespacedlowsmallcaps[1]{\spacedlowsmallcaps{\Large #1}}

\errorcontextlines=909

%% setups
\fnbelowfloat
\usepackage[final]{microtype}
\microtypesetup{babel,stretch=10,shrink=15,step=3,tracking=smallcaps}

\lstnewenvironment{code}[1][]%
{\lstset{#1}\relax%
}{%
}

%\usepackage{style}
\usepackage{blindtext}
\bibliographystyle{IEEEtran}

\begin{document}
\title{Live-Migration in virtuellen Umgebungen\\
  \large{Hypervisor - What have you done for me today?}
}
\author{Tim~Felgentreff und~Tobias~Pape%
\thanks{%
  Tim Felgentreff und Tobias Pape studieren am
  Hasso-Plattner-Institut, Potsdam.\goodbreak
  Email: \{vorname.nachname\}@student.hpi.uni-potsdam.de}%
}
\markboth{Industrieseminar Cloud-Computing}{Felgentreff, Pape: Live-Migration}

\maketitle
\tableofcontents

\begin{abstract}
  
\end{abstract}

% \begin{IEEEkeywords}
% \end{IEEEkeywords}
\IEEEpeerreviewmaketitle

\section{Einführung}
\label{sec:einfuehrung}
% \IEEEPARstart{H}{eutzutage} gibt es zu viele Probleme
% Today, many organizations deploy a multitude of services on a large
% number of servers. In some cases, all these services run on different
% machines not because one machine cannot handle the load, but for
% reasons of reliability.  All software is faulty~\cite{zellerprograms},
% and operating systems with thousands of lines of code, running servers
% written by different software vendors simply cannot be trusted to
% provide uptimes of 99.9 percent or more. Putting each service on a
% separate machine will achieve reliability at the expense of
% maintainability simply because a large number of physical machines is
% involved.

% Virtual machine technology has been around for more than 40
% years~\cite{tanenbaum1992modern}. The technology to host multiple
% machines on a single system reduces the space and power requirements
% for data centers and represents enormous cost savings for companies
% like Amazon, Microsoft or Google, which deploy hundreds or thousands
% of servers for a variety of services. Unsurprisingly, virtualization
% for cost reduction and transparent scalability is becoming
% increasingly popular. Cloud providers such as Amazon, Rackspace or
% Engine Yard offer scalable hosting solutions for companies and
% individuals who cannot of do not want to maintain a private cloud.

% There are obvious issues with isolation, however. As a service
% provider, I want to avoid downtime as much as possible. Thus, I cannot
% easily move my virtual machine from one provider to another. I cannot
% easily migrate existing VMs to new a hypervisor architecture or a
% completely new virtualization solution. If hardware is failing, I
% cannot simply migrate the VM to new hardware without taking it down.

% Asger Jensen and Dr. Jacob Gorm Hansen proposed live VM migration as a
% solution to these problems. They were the first to demonstrate live
% migration of a running VM in 2002 on a research hypervisor called
% NomadBIOS. Since then, HP has equipped it's HP-UX with live migration
% capabilities, VMWare has implemented it in its vMotion product, the
% open-source Xen project supports it and IBM is actively researching to
% enable VM migration on a number of open-source virtualization
% solutions.
- Firmen sehen heutzutage so aus
- Cloud könnte für sie interessant sein
- Welche Rolle Live-Migration da für sie spielt, und wie es ihnen
helfen kann, zeigen wir

\subsection{Mainframes und Cluster}
- Derzeit Rechner im Rechenzentrum
- Cluster
- Grid
- Probleme:
  - Ungleichmäßige Auslastung
  - Wartungskosten (Admins)
  - Downtime bei Upgrades/HW-Problemen
  - Operationskosten (Strom)

\subsection{Die Cloud als Ziel}
\label{sec:sota}
- Public vs Private
Für potentielle Nutzer einer Cloud Lösung präsentieren sich derzeit
zwei Möglichkeiten: Eine eigene, business-interne Cloud Plattform zu
installieren, oder 

In this part we will explore the state of the art in live VM
migration, both from a technical as from a use-case point of view.
Ich will scalieren:
\begin{itemize}
\item Public Cloud
\item Private Cloud
\end{itemize}
- Um zu entscheiden zunächst wissen, {\bf was} in die cloud geht

\subsubsection{Level der Virtualisierung}
\label{sec:def-virtualisierung}
\url{http://www.dcl.hpi.uni-potsdam.de/teaching/cloudIndustSem/slides/giesekus.pdf}
\begin{itemize}
\item Level der virualisierung
  \begin{itemize}
  \item hw (ldoms)
  \item hypervisoren !!
    
    Virtuelle Maschinen (VMs) laufen auf physikalischen Maschinen durch
    einen \emph{Hypervisor}, ein minimales Betriebssystem, dass den
    Zugriff auf die physikalische Hardware regelt und Routing zwischen dem
    physikalischen und virtuellen Netzwerk
    bereitstellt~\cite{tanenbaum1992modern}. Wenn eine VM ausfällt,
    bleiben die anderen davon unbeeinflusst. Da des weiteren alle
    Operationen innerhalb einer VM virtualiziert sind, kann der Hypervisor
    den Zustand eines virtuellen Servers in regelmäßigen Abständen
    sichern, und so bei Ausfall die VM von einer früheren Sicherung
    wiederherstellen.

  \item Os-partitioning (Solaris Zones)
  \item Process VMs (JVM etc.pp.)
  \end{itemize}
\end{itemize}

\subsubsection{Was geht in die Cloud?}
Auf welchen Ebenen gibt es "`Clouds"', welche Ebene betrachten wir,
PaaS vs IaaS.

\subsection{Neue Probleme in Clouds}
Probleme:
\begin{itemize}
\item Public
  \begin{itemize}
  \item Bindung an Provider
  \item Scalierbedarf zunächst unklar -> Kostenfrage
  \item Physikalischer Ort der Daten unbekannt -> VMs müssen in die
    private Cloud oder in Rechenzentren verschoben werden können ->
    gesetzl. Vorschriften, Intellectual Property, \ldots
  \end{itemize}
\item Private
  \begin{itemize}
  \item Scalierbedarf gegen Wartungsaufwand -> Menge der HW
  \item Gleichmäßige Auslastung zunächst genauso schwierig wie vorher
  \item Hardware-Wechsel soll nicht mehr zu Donwtime führen
  \end{itemize}
\end{itemize}

\section{Live-Migration als Teil der Lösung}
\label{sec:livemigration}
Dr. Jacob Gorm Hansen described the process of live migration
as implemented in his NomadBIOS, which we will recapitulate here.

The following sections summarize some of the real-world use cases for
live migration and how it compares to the previous solutions which
were applied to these problems.

\subsection{Verlässlichkeit von Virtualisierung}
In jedem verteilten System muss man mit der Realität umgehen, dass
Hardware und Software Fehler enthält, die zu Ausfällen führen
können. Da scheint die Idee, viele Virtuelle Maschinen auf wenigen
physikalischen Systemen auszuführen zunächst wie ein Schritt in die
falsche Richtung: Wenn dann ein physikalischer Server ausfällt, oder
die Virtualisierungssoftware einen Fehler enthält, fallen potentiell
eine ganze Reihe von Diensten auf einmal aus. In der Realität
entstehen die meisten Fehler in Datencentern allerdings durch
Softwarefehler~\cite{tanenbaum1992modern}. Softwarefehler können umso
häufiger zutage treten, je mehr unterschiedliche Systeme miteinander
interagieren müssen, und je größer die Code-Basis der einzelnen
Systeme ist~\cite{zellerprograms}. Durch Virtualisierung kann man
Softwaresysteme voneinander isolieren, ohne für jedes System einen
eigenen, physikalischen Server mit eigenem Betriebssystem
bereitzustellen. Der Hypervisor selbst ist in solch einem Fall die
einzige Software, die im Kernel Modus läuft, und dieser hat um
Größenordnungen weniger Zeilen Quelltext als ein aktuelles
Betriebssystem~\cite{tanenbaum1992modern}. Folglich enthält der
Hypervisor wahrscheinlich weniger Bugs~\cite{zellerprograms}.

Kritische Dienste können so in kompletter Isolation ausgeführt um das
Risiko eines Ausfalls durch inkompatible Software auf demselben Server
zu minimieren.

Nichtsdestotrotz müssen auch Vorkehrungen gegen Hardwarefehler
getroffen werden. In klassischen Hosting-Umgebungen konnte man bei
immanenten Ausfällen von physikalischen Systemen relevante Prozesse
auf neue VMs migrieren~\cite{hansen2004self}. Für eine gegebene VM
konnte eine Ähnliche (z.B. aus einem früheren, gesicherten Zustand)
auf einem neuen Server gestartet werden. Der Zustand der relevanten
Prozesse wurde soweit als möglich gemäß den Gegebenheiten des
Betriebssystems kopiert und neue Dienstanfragen wurden dann auf den
neuen Server umgeleitet. Die alte VM musste allerdings zunächst online
bleiben, um Abhängigkeiten von Dienstanfragen bereitzustellen, die
sich noch in Ausführung befanden, als der Hardwarealarm ausgelöst
wurde~\cite{clark2005live}.

\subsection{Interoparibilität von Virtualisierungs-Lösungen}
Several process migration systems were developed in the 1980s,
examples are Sprite and MOSIX\cite{hansen2004self}. Such migration
solutions depend on capabilities of the hypervisor as well as the
operating system, limiting their use across organization boundaries,
during system upgrades, as well as for cloud hosting solutions where
the OS software is not controlled by the hosting provider.

Virtualized machine hardware varies even today. There are efforts
underway\cite{cloudstandard} to standardize different aspects of
virtualization, like the format for virtual machine descriptions and
the APIs of hypervisors. This is a work-in-progress, however, and
moving complete systems between different vendors usually involves
shutting those systems down to allow various data conversions to take
place.

\subsection{Hardware-Aufstockung ohne Ausfallzeigen}
Modern businesses have to be able to react to market change quickly
and with minimal cost. New services are first explored on cheap
hardware\cite{tanenbaum1992modern}, but sudden bursts of customer
interest might require fast upscaling of the underlying
hardware. Being able to quickly scale up to rising interest is crucial
for the success of a service. Studies show that slow loading times
will cause people to use online services less or leave them and never
return\cite{kohavi2007online}. Historically, the quick success of such
a scenario dependent on quick and correct assertion of upcoming
hardware requirements and carefully planned downtime.

Using live migration, services can stay online and better hardware can
be booted in parallel to the existing. A new 

\subsection{Inter-Cloud Verschiebungen}
\label{sec:movclouds}
\subsection{Uptime Despite Hardware Failure}
\label{sec:hardfail}
\subsection{Scaling through Replication}
\label{sec:replication}
\begin{itemize}
\item Heroku
\item App engines auf anderem scale (IaaS)
\end{itemize}
\subsection{keine Ausfallzeit}
\label{sec:keine-ausfallzeit}
Netzwerklast is hoeher, aber who cares.
Yes: Github, heroku, engineyard.
\subsection{Adaptive Auslastung}
\label{sec:adaptive-auslastung}
SLAs zwingen nicht zum idlen herumsitzenlassen von Servern um
abzusichern, VMs können im Zweifel schnell umgezogen werden.
-> Contra: Durchsatz, Netzlast, Umzugszeit
-> Pro: Github uptime
\subsection{Langzeit-Lauffähigkeit}
\label{sec:langz-lauff}
Cloud nutzlos wenn der Server dann weiterhin alle paar Monate down
wegen:
\begin{itemize}
\item Hardware-Ausfalls
\item Hardware-Upgrades
\item Steigender Nachfrage
\end{itemize}

\section{Alternative Live-Migration-Systeme}
Ziele/Einsatzgebiete, Verbreitung

\subsection{IBM}
\subsection{HP}
\subsection{Sun/Solaris}

\section{Vergleich}
Vergleich der Eignung als Lösung für die oben genannten Probleme
zwischen dem von Dr. Hansen vorgestellen System und den Alternativen.

\section{Schlussfolgerungen, Zusammenfassung}
Überblick über die Bewertung der Systeme
Überblick über die Eignung für den Produktiveinsatz
Ausblick für weitere Entwicklung der VM-Verschiebung

% trigger a \newpage just before the given reference
% number - used to balance the columns on the last page
% adjust value as needed - may need to be readjusted if
% the document is modified later
%\IEEEtriggeratref{8}
% The "triggered" command can be changed if desired:
%\IEEEtriggercmd{\enlargethispage{-5in}}
\bibliography{IEEEabrv,Live-MigrationInVrituellenUmgebungen}
\end{document}

%%% Local Variables: 
%%% mode: latex
%%% TeX-master: t
%%% End: 
