\title{Live-Migration in virtuellen Umgebungen
%\\  \large{Hypervisor - What have you done for me today?}
}
\author{Tim~Felgentreff und~Tobias~Pape%
\thanks{%
  Die Autoren studieren am
  Hasso-Plattner-Institut, Potsdam.\goodbreak
  Email: \{vorname.nachname\}@student.hpi.uni-potsdam.de}%
}
\markboth{Industrieseminar Cloud-Computing}{Felgentreff, Pape: Live-Migration}

\maketitle

\onlydraft{\tableofcontents}

\begin{abstract}
\onlydraft{
\begin{itemize}
  \item Einf.
  \item Lifemig.
  \item Systeme
  \item Vgl.
  \end{itemize}
}
Beim Betrieb von Infrastrukturdiensten im Umfeld von Cloud-Computing
-- \acf{IaaS} -- gibt es
%neben vielen Vorteilen wie
%on-demand-Verfügbarkeit,
% auch
einige Herausforderungen wie Ausfallzeitminimierung,
Hardware-Änderungen im laufenden Betrieb und Skaliereung nach Last.
Mit Hilfe von Live-Migration von virtuellen Maschinen ist es möglich,
einige davon anzugehen. Es gibt diverse
Virtualisierungs-Platformen unterschiedlicher Hersteller, die
Live-Migration auch zu einem hohen Grad unterstützen. Es mangelt noch
an Interoperabilität zwischen den Systemen und Standardisierung ist
in diesem Bereich noch nicht weit fortgeschritten.
\end{abstract}

\begin{IEEEkeywords}
  Cloud-Computing, Virtualisierung, Live-Migration, IaaS, Interoperabilität
\end{IEEEkeywords}
\IEEEpeerreviewmaketitle

%%% Local Variables:
%%% mode: latex
%%% TeX-master: "FelgentreffPape_2010_Live-MigrationInVirtuellenUmgebungen"
%%% End:
