\documentclass{scrartcl}

%\let\ifdebug\iftrue
\let\ifdebug\iffalse

\usepackage{style}
\usepackage{blindtext}
\bibliographystyle{alpha}

\title{Live VM Migration technology for Cloud hosting providers}
\subtitle{Ab inne cloud}
%\authoremail{Tim Felgentreff \and Tobias Pape}{\{firstname.lastname\}@student.hpi.uni-potsdam.de}
\author{Tim Felgentreff \and Tobias Pape}

\begin{document}
\maketitle
\tableofcontents

\section{Introduction}
\label{sec:introduction}
Today, many organizations deploy a multiude of services on a large
number of servers. In some cases, all these services run on different
machines not because one machine cannot handle the load, but for
reasons of reliability.  All software is faulty\cite{Zeller2006}, and
operating systems with thousands of lines of code, running servers
written by different software vendors simply cannot be trusted to
provide uptimes of 99.9 percent or more. Putting each service on a
separate machine will achieve reliablity at the expense of
maintainability simply because a large number of physical machines is
involved.

Virtual machine technology has been around for more than 40
years\cite{tanenbaum1992modern}. The technology to host multiple
machines on a single system reduces the space and power requirements
for data centers and represents enormous cost savings for companies
like Amazon, Microsoft or Google, which deploy hundreds or thousands
of servers for a variety of services. Unsurprisingly, virtualization
for cost reduction and transparent scalability is becoming
increasingly popular. Cloud providers such as Amazon, Rackspace or
Engine Yard offer scalable hosting solutions for companies and
individiuals who cannot of do not want to maintain a private cloud.

There are obvious issues with isolation, however. As a service
provider, I want to avoid downtime as much as possible. Thus, I cannot
easily move my virtual machine from one provider to another. I cannot
easily migrate existing VMs to new a hypervisor architecture or a
completely new virtualization solution. If hardware is failing, I
cannot simply migrate the VM to new hardware without taking it down.

Asger Jensen and Dr. Jacob Gorm Hansen proposed live VM migration as a
solution to these problems. They were the first to demonstrate live
migration of a running VM in 2002 on a research hypervisor called
NomadBIOS. Since then, HP has equipped it's HP-UX with live migration
capabilities, VMWare has implemented it in its vMotion product, the
open-source Xen project supports it and IBM is actively researching to
enable VM migration on a number of open-source virtualization
solutions.

\section{State of the Art before Live Migration}
\subsection{Reliability of Virtualization}
\label{sec:sota}

Running a large number of virtual machines (VMs) on very few physical
nodes seems like betting all your money on one horse only. If one
physical node fails a number of services will be unavailable all at
once, probably much more than if a dedicated machine would
crash. However, most server failures in data centers are due to buggy
software, especially operating systems and servers, not hardware
failure. Virtualization isolates such failures, because VMs run in an
isolated environment atop a \emph{hypervisor}.

A hypervisor controls access to the physical hardware, provides
routing from the physical network into the virtual network and
isolates machines from one another. If any one machine should fail,
all others will be unaffected. Because all of the operations within a
VM are virtualized, the hypervisor can checkpoint virtual servers at
regular intervals. Should a VM fail, it can be restored from an
earlier checkpoint and resume operations.

Critical services can be run in complete isolation to minimize the
potential for failure due to faulty or incompatible software
stacks. The hypervisor itself is the only software running in kernel
mode on the physical machine and has orders of magnitude fewer lines
of code than the average operating system. This means it is equally
less likely to encounter a bug in it's programming.

Still, hardware failures, while less likely than software bugs, are an
issue that has to be considered. Before live migration of VMs was
possible, an alarm state from a hardware monitor could be answered by
migrating the relevant processes to new VMs\cite{hansen2004self}. For
a given VM, another one similar to it would be booted on a another
machine. The state of the services could be copied and new service
requests would be routed to the fresh VM, with the old VM only
finishing in-progress requests before shutting down.

\subsection{Compatibility of Virtualization Solutions}
Several process migration systems were developed in the 1980s,
examples are Sprite and MOSIX\cite{hansen2004self}. Such migration
solutions depend on capabilities of the hypervisor as well as the
operating system, limiting their use across organization boundaries,
during system upgrades, as well as for cloud hosting solutions where
the OS software is not controlled by the hosting provider.



\subsection{Hardware Upgrades without Downtime}

\section{Live Migration}
\label{sec:livemigration}

\blindtext

\subsection{Moving between Clouds}
\label{sec:movclouds}

\blindtext

\ParSep

\blindtext

\subsection{Uptime Despite Hardware Failure}
\label{sec:hardfail}

Ble

\subsection{Scaling through Replication}

\begin{code}[language=SQL]
select  * from foo where true;
\end{code}
\spacedlowsmallcaps{TheFooTest}
\spacedallcaps{test}
\blinddocument

%\bibliography{timfelgentreff}
\end{document}

%%% Local Variables: 
%%% mode: latex
%%% TeX-master: t
%%% End: 
